% ===
%
% Official LaTeX beamer template of 
% Chair for AI Methodology (AIM)
% RWTH Aachen University, Aachen, Germany
%
% Author: Jakob Bossek (bossek@aim.rwth-aachen.de)
% Based on earlier presentation slide templates by 
% Jakob Bossek and Holger H. Hoos.
% 
% AIM website: https://aim.rwth-aachen.de/
%
% ===

% HINT: add option aspectratio=169 for 16:9 aspect ratio
% HINT: add 'handout' to activate handout mode (without overlays for printing)
\documentclass[t,english]{beamer}

% include package import and macro defintions
% LOAD ESSENTIAL PACKAGES
% ===

\usepackage[english]{babel}

% Handling bibliography
\usepackage[backend=biber,sorting=none,style=authoryear]{biblatex}
\usepackage[hang]{footmisc}

% Math
\usepackage{amsmath}
\usepackage{upgreek}

% Miscelaneous
\usepackage{pgfpages}
\usepackage{csquotes}
\usepackage{placeins}

% Captions
\usepackage[labelformat=empty]{caption}

% Colors
\usepackage{xcolor}

% Requirements
\RequirePackage{tikz}
\usetikzlibrary{calc}
\usetikzlibrary{arrows,shapes,positioning}

% For \smiley{} and \frowney{}
\usepackage{wasysym}

% Pseudocode
\usepackage[vlined, linesnumbered, ruled]{algorithm2e}
\SetKwComment{Comment}{/* }{*/}
\SetKwInput{KwData}{Input}
\SetKwInput{KwResult}{Output}
\IncMargin{1.5em}

% nice icons
\usepackage{fontawesome}

% IMPORTANT NOTE: outputdir needs to be equal to output/aux dir of xelatex
%\usepackage[outputdir=output]{minted}
\usepackage{appendixnumberbeamer}

% Tables
\usepackage{booktabs}
\usepackage{multicol}
\usepackage{multirow}
\usepackage{colortbl}

% Date formatting
\usepackage[short,nodayofweek]{datetime}

% Hyperlinks
\RequirePackage{hyperref}
\hypersetup{colorlinks,linkcolor={rwth-blue-50},citecolor={rwth-black-75},urlcolor={rwth-blue-100}}
\let\oldciteauthor=\citeauthor
\def\citeauthor#1{\hypersetup{citecolor=rwth-black-50}\oldciteauthor{#1}}

\input{includes/rgb}
%\newcommand{\cmnt}[1]{{\footnotesize\bf #1}}
\newcommand{\cmnt}[1]{}

\newcommand{\ie}{{\it{}i.e.\/}}
\newcommand{\eg}{{\it{}e.g.\/}}
\newcommand{\cf}{{\it{}cf.\/}}
\newcommand{\wrt}{\mbox{w.r.t.}}
\newcommand{\vs}{{\it{}vs\/}}
\newcommand{\vsp}{{\it{}vs\/}}
\newcommand{\etc}{{\it{}etc.}}
\newcommand{\etal}{{\it{}et al.\/}}

\newcommand{\cemph}[2]{\emph{\textcolor{#1}{#2}}}
%\definecolor{purple}{rgb}{1,0,1}
%\definecolor{fuchsia}{rgb}{0.9,0.0,0.3}
%\definecolor{dred}{rgb}{0.7,0,0}
\definecolor{dblue}{rgb}{0,0,0.7}
\definecolor{dgreen}{rgb}{0,0.7,0}
\definecolor{dgrey}{rgb}{0.5,0.5,0.5}
\definecolor{grey}{rgb}{0.75,0.75,0.75}
\definecolor{teal}{rgb}{0,0.7,0.7}


\newcommand{\pscProc}{{\bf procedure}}
\newcommand{\pscBegin}{{\bf begin}}
\newcommand{\pscEnd}{{\bf end}}
\newcommand{\pscEndIf}{{\bf endif}}
\newcommand{\pscFor}{{\bf for}}
\newcommand{\pscEach}{{\bf each}}
\newcommand{\pscThen}{{\bf then}}
\newcommand{\pscElse}{{\bf else}}
\newcommand{\pscWhile}{{\bf while}}
\newcommand{\pscIf}{{\bf if}}
\newcommand{\pscRepeat}{{\bf repeat}}
\newcommand{\pscUntil}{{\bf until}}
\newcommand{\pscWithProb}{{\bf with probability}}
\newcommand{\pscOtherwise}{{\bf otherwise}}
\newcommand{\pscDo}{{\bf do}}
\newcommand{\pscTo}{{\bf to}}
\newcommand{\pscOr}{{\bf or}}
\newcommand{\pscAnd}{{\bf and}}
\newcommand{\pscNot}{{\bf not}}
\newcommand{\pscFalse}{{\bf false}}
\newcommand{\pscEachElOf}{{\bf each element of}}
\newcommand{\pscReturn}{{\bf return}}

\newcommand{\param}[1]{{\sl{}#1}}
\newcommand{\var}[1]{{\it{}#1}}
\newcommand{\cond}[1]{{\sf{}#1}}
\newcommand{\state}[1]{{\sf{}#1}}
\newcommand{\func}[1]{{\sl{}#1}}
\newcommand{\set}[1]{{\Bbb #1}}
\newcommand{\inst}[1]{{\tt{}#1}}
\newcommand{\myurl}[1]{{\small\sf #1}}

\newcommand{\Nats}{{\Bbb N}}
\newcommand{\Reals}{{\Bbb R}}
\newcommand{\extset}[2]{\{#1 \; | \; #2\}}

\newcommand{\vbar}{$\,\;|$\hspace*{-1em}\raisebox{-0.3mm}{$\,\;\;|$}}
\newcommand{\vendbar}{\raisebox{+0.4mm}{$\,\;|$}}
\newcommand{\vend}{$\,\:\lfloor$}

\newcommand{\OR}{\mbox{ or }}
\newcommand{\AND}{\mbox{ and }}
\newcommand{\NOT}{\mbox{not }}
\newcommand{\VAR}[1]{x_{#1}}

\newcommand{\CNP}{{\cal NP}}
\newcommand{\CP}{{\cal P}}

%% commenting macros

% use this to hide larger blocks of material:
\usepackage{xcolor}
\usepackage{amsmath,amssymb}

% Define colors for authors
\definecolor{janecolor}{rgb}{0.2,0.6,0.6}
\definecolor{johncolor}{rgb}{0,0.7,0}

% Define colors for general macros
\definecolor{todocolor}{rgb}{0.9,0.1,0.1}
\definecolor{changedcolor}{rgb}{0.42,0.27,0.57}
\definecolor{addedcolor}{rgb}{0.867,0.176,0.361}

% General comment macro
\newcommand{\nbc}[3]{
    {\colorbox{#3}{\bfseries\sffamily\scriptsize\textcolor{white}{#1}}}
    {\textcolor{#3}{\sf\small$\blacktriangleright$\textit{#2}$\blacktriangleleft$}}
}
  
% Define individual comments for authors
\newcommand{\jane}[1]{\nbc{Jane}{#1}{janecolor}}
\newcommand{\john}[1]{\nbc{John}{#1}{johncolor}}

% Define general helper macros
\newcommand{\todo}[1]{\nbc{TODO}{#1}{todocolor}}
\newcommand{\changed}[1]{\nbc{CHANGED}{#1}{changedcolor}}
\newcommand{\added}[1]{\nbc{ADDED}{#1}{addedcolor}}

\newcommand{\redacted}[1]{\emph{[anonymized for review]}}
%\renewcommand{\redacted}[1]{#1}

% Use this to temporarily hide reviewing comments, todos, etc.:
%  \renewcommand{\jane}[1]{}
%  \renewcommand{\john}[1]{}

% Use this to make "changed" items appear normal:
%\renewcommand{\changed}[1]{#1}
%\renewcommand{\added}[1]{#1}
%\renewcommand{\todo}[1]{}

%% end commenting macros


% add (multiple) bibliography sources for biblatex
\addbibresource{bib.bib}

% NOTE: the predefined options are best practices of AIM!
%       Do not change for seminar talks!
% Set authorinfo={0,1} to (de)activate short author and short title in footer
% Set progress={0,1} to switch between x and x/total display of pages in the bottom right corner
% Set outline={0,1} to (de)activate outline slides at the beginning of each section
\usetheme[authorinfo=1, progress=0, outline=0]{AIM}

% metadata
\title[The AIM \LaTeX{} Beamer Template]{Official \LaTeX{} Beamer Template \newline of the Chair for AI Methodology~(\textbf{AIM}) 
\newline 
RWTH Aachen University}
\subtitle{Quick Guide}
\author[Bossek \& Hoos]{\underline{Jakob Bossek}\inst{1} \and Holger H. Hoos\inst{1,2}}
\institute{
\inst{1}Dept. of Computer Science, RWTH Aachen University, Germany\\
\inst{2}LIACS, Universiteit Leiden, The Netherlands
}
\date{\small\today}

% optional
\titlegraphic{
  \includegraphics[width=2cm]{figures/rwth-logo.png}
}

%%% NOTE: do not remove the following marker. We use it for automatic compilation of multiple versions
%%% of the slides (e.g., handout, presentation with overlays) via compile.py
%%%pythonmarker

\begin{document}

\begin{frame}[plain]
\titlepage
\end{frame}

\addtocounter{framenumber}{-1}

% BASICS
% ======

\begin{frame}
  \frametitle{Introduction}
  
  \begin{itemize}
      \item Please take the time to briefly go over the instructions in this example presentation. 
      In particular, how to cite correctly.
      Make sure that your slides are overall not too busy $\rightarrow$ tables and figures should not contain too much information and you should not overwhelm listeners with walls of text. 
      \item Use an introductory slide (or introductory slides) to motivate the topic and to raise interest.
      \item You can show an outline of the talk (table of contents) if you want, but we strongly recommend to place it after the introduction and \emph{not} before.
  \end{itemize}
    
\end{frame}


\section{Basics}

\begin{frame}
  \frametitle{Blocks}

  \begin{block}{Regular block}
    This is a plain and simple block.
  \end{block}

  \begin{exampleblock}{Example block}
    This is for examples.
  \end{exampleblock}

  \begin{alertblock}{Alert block}
    Use this one to state important information.
  \end{alertblock}

\end{frame}

\begin{frame}
  \frametitle{Lists}

  \begin{itemize}[<+->]
    \item Lorem ipsum ...
    \item dolor sit amet ...
    \begin{itemize}
      \item Consequetur amibilisque utero
      \item Anhilore deus et arendum
    \end{itemize}
    \item[$\leadsto$] Custom label
    \begin{enumerate}
      \item Idiquit et collequt deribur
      \item Canum meum id comedid
    \end{enumerate}
  \end{itemize}

\end{frame}

\begin{frame}[fragile]
  \frametitle{Citations with biblatex}
  \framesubtitle{Good alternative to natbib}

  Sample citation:~\cite{BNPS2019}\\
  Sample citation in parenthesis:~\parencite{BNPS2019}\\
  Sample full citation via \verb|\fullcite|\\[0.5cm]
  \fullcite{BNPS2019}.\\[0.5cm]

  \bigskip

  Only use full citations if \textit{really} necessary, otherwise full citations should only be shown at the end of the presentation.

\end{frame}

\begin{frame}[fragile]
  \frametitle{Images}

  Use macro \verb|\ig{width}{path-to-file}| for a single \textbf{centered} image:
  \ig{0.4}{example-image-a}
\end{frame}


\begin{frame}
  \frametitle{Figure environment}

  Feel free to use the figure environment. Note that in the captions \textbf{(also for tables)} our template \textit{deliberately} omits the label 'Figure:' before the caption (everyone can see that it is a Figure or a Table):

  \begin{figure}[H]
    \includegraphics[width=0.5\textwidth]{example-image-a}
    \caption{test}
  \end{figure}

\end{frame}

\begin{frame}
  \frametitle{Tables}

  We recommend to take a look at the presentation \href{https://people.inf.ethz.ch/markusp/teaching/guides/guide-tables.pdf}{Small Guide to Making Nice Tables} by Markus P\"uschel.
  Note that in the captions \textbf{(also for figures)} our template \textit{deliberately} omits the label 'Table:' before the caption (everyone can see that it is a Figure or a Table):

  \begin{center}
    \renewcommand{\tabcolsep}{4pt}
    \renewcommand{\arraystretch}{1.1}
    \begin{footnotesize}
    \begin{tabular}{ccrrrrrr}
    \toprule
    \multirow{2}{*}{\textbf{TSP Set}} & \multirow{2}{*}{\textbf{Mutation}} & \multicolumn{2}{c}{\bfseries RTS\textsuperscript{*}} & \multicolumn{2}{c}{\bfseries FR\textsuperscript{\dag}} & \multicolumn{2}{c}{\bfseries PAR10} \\
    \cmidrule(l{2pt}r{2pt}){3-4} \cmidrule(l{2pt}r{2pt}){5-6} \cmidrule(l{2pt}r{2pt}){7-8}
     &  & \textbf{EAX} & \textbf{LKH} & \textbf{EAX} & \textbf{LKH} & \textbf{EAX} & \textbf{LKH}\\
    \midrule
    RUE & - & 1.26 & 0.74 & 0.00 & 0.00 & 1.26 & 0.74\\
    \midrule
    Easy for & simple & 1.34 & 912.78 & 0.00 & 0.20 & 1.34 & 7\,608.11\\
    \cmidrule{2-8}
    EAX & sophistic. & 0.97 & 830.80 & 0.00 & 0.22 & 0.97 & 8\,230.61\\
    \midrule
    Easy for & simple & 117.97 & 0.74 & 0.00 & 0.00 & 117.97 & 0.74\\
    \cmidrule{2-8}
    LKH & sophistic. & 67.90 & 0.88 & 0.00 & 0.00 & 67.90 & 0.88\\
    \bottomrule
    \multicolumn{8}{l}{\textsuperscript{*} \tiny RTS: Running time of successful runs, \textsuperscript{\dag} FR: Failure ratio}\\%
    \end{tabular}
    \end{footnotesize}
  \end{center}

\end{frame}

\begin{frame}[fragile]
  \frametitle{Columns}

  The \verb|\twocol{...}{...}|-macro aligns stuff on top and uses two equally sized columns:
  \twocol{
    \begin{itemize}
      \item Lorem ipsum
      \item dolor sit amet
      \item consequetur deribilis auret
      \begin{itemize}
        \item Alamat deceductovo ameritol
        \item Consequencias pavit
      \end{itemize}
    \end{itemize}
  }{
    \vskip-15pt
    \ig{1}{example-image-b}
  }
\end{frame}


% MATHEMATICS
% ===========

\begin{frame}
  \frametitle{Math}
  \framesubtitle{... looks awesome in \LaTeX{}}

  \begin{block}{Sample formula}
    Math looks so \alert<2->{awesome} in \LaTeX{}!

    \begin{align*}
      \hat{\theta}_{\text{ML}} = T(X_1, \ldots, X_n) = \frac{n}{n-1} \sum_{\frac{i=1}{i \neq k}}^{m} \left(X_i^2 - \exp(X_i - X_k)\right)^{k/2}
    \end{align*}
  \end{block}

  \begin{theorem}[\cite{BNPS2019}]
    For every tree with $n$ nodes and maximum degree $\Delta$ the expected time until RLS and (1+1) EA find an optimal $\Delta$-edge-coloring is $O(\Delta \ell^2 m \log m)$ where $\ell$ is the length of the longest path in the tree.
  \end{theorem}
\end{frame}


% SOURCE CODE
% ===========

\begin{frame}[fragile]
  \frametitle{Pseudo-code}
  \framesubtitle{... with \texttt{algorithm2e}}

  % NOTE: H is necessary as well as the fragile option for the frame
  \begin{algorithm}[H]
  \caption{Sample algorithm (taken from Overleaf docs)}\label{alg:sample_algorithm}
  \KwData{$n \geq 0$}
  \KwResult{$y = x^n$}
  $y \gets 1$, $X \gets x$, $N \gets n$\;
  \While{$N \neq 0$}{
    \eIf{$N$ is even}{
      $X \gets X \times X$\;
      $N \gets \frac{N}{2}$\tcp*[r]{This is a comment!}
    }{\If{$N$ is odd}{
        $y \gets y \times X$\;
        $N \gets N - 1$\;
      }
    }
  }
  \end{algorithm}

\end{frame}

% PREDEFINED MACROS
% =================

\section{Predefined macros}

\begin{frame}[fragile]
  \frametitle{Useful macros}

  \begin{itemize}
    \item Use \verb|\refer{...}| to refer a paper quickly (without the need for a bibfile entry): \refer{Bossek et al., 2019}
    \item Use \verb|\hide{...}| to temporarily hide block of code.
    \item Use \verb|\plainframe{...}| for a visually reduced frame with horizontally and vertically centered message/image (see next slides for examples).
    \item Use \verb|\plainframe[mycolor]{...}| to change the default background.
  \end{itemize}

\end{frame}

% no header, just some centred text
\plainframe{\textcolor{rwth-blue-100}{Plain slide (focus on certain element)}}

% set specific background color with an optional argument
\plainframe[black]{\includegraphics[width=0.5\textwidth]{example-image-a}}

\begin{frame}[fragile]
  \frametitle{Commenting macros}
    
  It is often useful to comment on different things while writing a report, adding ToDos or highlight changed or added parts. To this end the file \texttt{includes/commenting.tex} defines some useful macros.
  \begin{itemize}
    \item Use \verb|\todo{...}| to add a ToDo:\newline \todo{Do this, do that}
    \item Use \verb|\changed{...}| to indicate changes:\newline \changed{This text was changed.}
    \item Use \verb|\added{...}| to highlight additions:\newline \added{This text was added.}
    \item Use author-specific macros, e.g., \verb|\jane{...}| or \verb|\john| for our two sample authors Jane and Joe, to add comments. Feel free to edit \texttt{includes/commenting.tex} to add/adapt the author-specific macros.\newline{}
    \jane{Comment by Jane.}\newline
    \john{Comment by Joe.}
  \end{itemize}
    
\end{frame}

% CONCLUSION
% ==========

\begin{frame}
  \frametitle{Take-home message}
  
  \begin{itemize}
      \item Briefly summarise the main finding. What are the most important aspects the audience should keep in mind?
      \item Your presentation should never end with a slide showing the references or a plain slide with \emph{Questions?} I.e., our recommendation is to end the presentation with the take-home messages.
  \end{itemize}
\end{frame}

\appendix

% REFERENCES
% ==========

\begin{frame}[allowframebreaks]
  \frametitle{References}
  \printbibliography[heading=none]
\end{frame}

% BACKUP SLIDES
% =============

\begin{frame}
  \frametitle{Backup slides}

  Sequence of additional slides. Useful to keep more information in the background which can be revealed during discussions.

\end{frame}

\end{document}
